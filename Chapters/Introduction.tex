

\section{Expert behaviour}
\label{xbehaviour}

An expert in a task domain can think effectively about problems in that
domain. It is not only the general abilities such as memory and
intelligence or talent that lead to expertise, rather extensive practice
and exploration of the task domain are the more important factors
\citep{ericsson02}. The knowledge acquired during the exploration of the
problem domain enables the expert to perceive and understand the same
problem in a more effective way than a novice.

Although a novice and an expert may possess the same amount of relevant
knowledge in terms of facts and formulae, the superior abilities of experts
arise from the organization of the knowledge that happens as a result of
experience. Experts are able to retrieve important aspects of their
knowledge with little attention or effort. Their knowledge isn't just a
collection of facts and propositions, instead it reflects the context of
applicability, that is, knowledge is ``conditionalized'' on a set of
circumstances \citep{glaser99, bransford00}. For example an expert chess
player is able to recall meaningful board positions much more accurately
than average players, while recall ability for random board positions are
similar for both experts and non-experts \citep{degroot65, chase73}. These
evidences suggest that with experience and practice, people cognitively
chunk the available information in their task domain into larger units
based on their functional characteristics. The evidence for knowledge
chunking in experts has been observed in many fields, e.g. in the game of
bridge \citep{engle78}, in electronic circuitry \citep{egan79} and computer
programming \citep{ehrlich1984}.

A key aspect in chunking is abstraction. An expert is able to conceptualize
the most {\em important} principles pertinent to solution to a problem and
ignore the rest.  Chunks are formed based on the functional relation of the
individual elements, ignoring other less important issues. For example in
chess, given a valid board position, expert players are more likely to
recall only a set of ``good'' moves for the position, out of a very large
number of possible moves.


Although design problems are different from problems in other domains
\citep{schon88, goel92}, expert designers too rely on their experiential
knowledge to guide the design process. The experience helps the designers
to frame the problem and arrive at a principal solution early on, and
identify the aspects of the design problem that need attention. Later
they are able to change the early solution rather fluently and easily as
difficulties are encountered.

\subsection{A real design situation}
\label{expertInAction}
In this work, we are considering four design problems. For one of these,
the brush-less DC permanent magnate motor (BDCPMM) design problem, we were
able to talk to Dr. S. P. Das who has expertise in brush-less DC motors. The
problem is to design custom motors using off-the-shelf standard parts. The
two objectives of the design are to minimize the production cost and to
maximize the peak torque. The expression for torque produced in a BDCPMM is
\begin{align}
T_p = 87300 C_{tor} N R_{si} A_{wire} n_{l} 
\end{align}
where $C_{tor}$ is either $\frac{1}{3}$ or $\frac{2}{3}$ for $Y$ connection
and $\Delta$ connection respectively, $N$ is the number of laminations in
the stator, $R_{si}$ is the inner radius of the stator, $A_{wire}$ is the
cross-sectional area of the wire used in the stator winding and $n_l$ is
the number of turns in the stator coil.  For low cost motors, he said that
without any other constraints, designs with smaller radial dimensions and
thin wires for winding would be cost effective. For high end motors,
however, a design with larger radial dimensions would be better in terms of
cost, though the choice of wire for winding would require investigation.
In both the cases similar designs would tend to vary considerably in the
number of laminations in the stator. On the question of the connection
type for the windings he could immediately choose the $\Delta$ connection,
though the reasons for this choice were not so clear. We were surprised to
note that each of these observations directly follow from our results and
are discussed in section \ref{bdcpmDiscuss}.


% When the problem of designing a
% motor with a torque specification was presented to an expert, he said that
% without any other constraints, designs with laminations with smallest
% radial dimenions and thin wires for winding would be cost effective for low
% torque motors. For high end motors, however, a design with laminations with
% larger radial dimensions would be better in terms of cost, though the
% choice of wire for winding would require investigation.


\section{Modelling the chunking behaviour}
The endeavour to computationally model the human learning behaviour dates
back several decades. First attempts at modelling the chunking behaviour
such as the EPAM or CHREST \citep{feigen61, gobet93} tried pure and direct
implementation of chunking mechanisms based on discrimination networks. An
alternative approach employed in systems such as the Soar and ACT-R
\citep{anderson96} relies on a production-rule based representation of the
information. A further alternative is presented by non-symbolic
computational models \citep{elman05}, which propose that information should
be distributed across multiple units, and no units should be dedicated to
specific functions.

In the past automated design systems such as CADET \citep{sycara91} and
ARCHIE-II \citep{domeshek91} have focused on using the knowledge stored as
cases in new design problems, however they are limited in their ability to
incorporate new knowledge from design experience into their knowledge
base. Attempts to incorporate knowledge transfer in search oriented design
systems have bean made \citep{moss04}, however the learning is in terms of
concrete design components, and carries no knowledge of the context of
application. A methodology for automatically extracting innovative implicit
principles of optimal design from multi-objective optimization has been
proposed in \citep{deb10}. In \citep{mukerjee09} it is argued that human
flexibility in symbol usage is possible because they are grounded on
experience and any attempt at concrete definition will miss many of the
associations the symbol has. This work is based on this view, and we
propose the {\em chunk dimensionality conjecture} in the next section and
present the results of empirical work on this conjecture in the next few
chapters.

\section{Chunk dimensionality conjecture}
\label{cdc}

In many cases it has been observed that the dimensionality of the optimal
solution set manifold in the decision space is the same as its
dimensionality in the objective space \citep{mukerjee09}. In this work we
have tested five multi-objective optimization tasks and found that the
pareto-front for these problems could be divided into clusters, most of
which had manifold dimensionality comparable to the number of
objectives. This evidence led us to propose the following conjecture, which
we will not attempt to prove in this thesis since the constraints that the
conjecture imposes are not met by most practical multi-objective
optimization situations, but we will present the empirical evidence that
substantiate this conjecture.


\begin{conjecture}[Chunk dimensionality conjecture]\label{cdconjecture}
  Given a multi-objective optimization problem with a decision variable $x
  \in \Sigma$, where $\Sigma$ is the decision space,

% For
%   $d$-objective multi-objective optimization problem with a $D$ dimensional
%   decision space,

  \begin{enumerate}[(a)]
    \item chunks emerge from a high-dimensional decision space
      $\mathbb{R}^{D}$ as clusters among the better performing combinations,
    \item chunks reflect a lower dimensionality than the embedding space,
      i.e.  chunks are manifolds of dimension $d_{c}$, $d_{c} < D$,
    \item for multi-objective decision problems with $d+1$ objectives ($d
      \ll D$), the better performing combinations are to be found on the
      non-dominated (pareto) frontier which is a $d$-dimensional manifold
      in the objective space, and
    \item if the objective function that maps the decision space to the
      objective space is continuous and well-behaved, this would result in
      chunks that have a dimensionality $d_{c} = \textbf{o}(d)$,
      i.e. $d_{c} = kd$, where $k-1$ is vanishingly small.
  \end{enumerate}
\end{conjecture}

Claims a, b, and c of the conjecture are fairly well accepted, d requires
further comment. An objective function $f$ is a mapping from decision space
to objective space. Let $\Phi \subset \mathbb{R}^{d+1}$ be the objective
space then assuming $\Sigma$ and $\Phi$ are metric spaces, an objective function $f$ is a the mapping: \\
$f: \Sigma \mapsto \Phi$ \\
The vector function $f$ is continuous if and only if each component
(scalar) function is continuous. A component function $g$ of $f$ is
continuous iff:\\
$\forall x, x' \in \Sigma$ and $\forall \epsilon \in \mathbb{R}_{\geqslant 0}$, ;\\
$\exists \delta \in \mathbb{R}_{\geqslant 0}$ , such that \\
$|x - x'| < \delta \Rightarrow |g(x) - g(x')| < \epsilon$\\
i.e. two neighboring points $x$ and $x'$ in the decision space will map to
$y$ and $y'$ in the objective space, which are also {\em near} in the
objective space. Local neighborhood in the objective space may reflect
either a local neighborhood in the decision space, or a combination of such
neighborhoods. By well behaved we imply that the function does not
oscillate very wildly or exhibit other such undesirable behaviour.

In some optimization tasks, the objective mapping function is extremely ill
posed and such constraints may not hold. However in many common
optimization tasks, the objective functions are quite well behaved. Table
\ref{expSummary} shows that results from practical multi-objective
optimization support the conjecture. In most cases at most some of the 
decision variables are continuous. Despite this, we can observe that for
all the problems most of the clusters have a dimensionality $d_c = d$.

\begin{table}[!ht]
  \centering
  \begin{tabular}{|c|c|c|c|c|c|}
    \hline
    \multirow{2}{*}{Design problem} & \multirow{2}{*}{$D$} & \multirow{2}{*}{Cont. variables} & \multirow{2}{*}{$d+1$}  & \multicolumn{2}{c|}{Cluster dimensionality} \\
    \cline{5-6} 
    &&&&Dimensionality & No. of clusters \\
    \hline
    \multirow{2}{*}{BDCPM design} & \multirow{2}{*}{5} & \multirow{2}{*}{0} & \multirow{2}{*}{2} & 1 & 4 \\
    \cline{5-6} 
    &&&&2 & 1 \\
    \hline
    Gearbox design (A) & 11 &10 & 2 & 1 & 11 \\
    \hline
    Gearbox design (B) & 29 & 10 & 3 & 2 & 7 \\
    \hline
    Clutch brake design & 5 & 0 & 2 & 1 & 5 \\
    \hline
    Welded beam design & 4 & 4 & 2 & 1 & 5 \\

    % &&& 2 & 1\\
    \hline
    % \multicolumn{1}{|c|}{\multirow{2}{*}{\textbf{C}}} & 0.435 & 0.897 & -0.066 & -0.003 \\ \cline{2-5}
    % \multicolumn{1}{|c|}{}& 0 & 0.006 & 0.04 & 0.999\\
    % \hline
  \end{tabular}
  \caption{Summary of experiments.}
  \label{expSummary}
\end{table}


\section{Dimensionality reduction}
A variety of dimensionality reduction techniques are employed to find low
dimensional structures \citep{mph07}. Linear dimensionality reduction
techniques assume a linear relationship among the points in the data and
try to find a low dimensional embedding for the data-set, while non-linear
dimensionality reduction techniques assume that the points in the data-set
lie along a manifold in the high dimensional space and try to find a low
dimensional embedding for this manifold. For non-linear data-sets, the
linear algorithms map smallest convex subspaces encapsulating the manifold,
often of much higher dimension than the manifold itself. On the other hand,
non-linear techniques embed the data-set in the manifold dimensions, whose
relation to the original input dimensions may be difficult to unravel.

Principal component analysis (PCA) \citep{hotelling33, jolliffe02}is by far
the most popular unsupervised linear dimensionality reduction technique.
PCA constructs a low dimensional representation of the data that describes
as much of the variance in the data as possible. This is done by finding a
linear basis of reduced dimensionality for the data using singular value
decomposition techniques, in which the amount of variance in the data is
maximal. The amount of variance along a reduced dimension is its {\em
  explained variance}. We use PCA to uncover the implicit common principles
that the designs in a chunk share, and to obtain a low dimensional
representations of the chunks.

Nonlinear dimensionality reduction techniques fall in one of two
categories: (1) techniques that attempt to preserve the local properties of
the original data in the reduced dimensionality representation, and, (2)
techniques that aim to preserve the global properties of the data in the
low-dimensional representation. {\em Locally linear embedding} (LLE)
\citep{saul03} is a major representative of local techniques. LLE models
the manifold as a collection of linear patches and attempts to characterize
the geometry of these linear patches. To do so it attempts to represent
each point $x_i$ as a weighted, convex combination of its $k$ nearest
neighbors. A $d$ dimensional configuration is then found whose local
geometry is characterized well the weight matrix obtained earlier. LLE has
been used in \citep{mukerjee09} to obtain a low dimensional representation
for chunks of {\em good} designs.

Isometric feature mapping or Isomap \citep{tsl05} is a global technique for
manifold learning that first captures the intrinsic geometry of the data in
geodesic manifold distances between all pairs of data. Geodesic distances
are estimated by first constructing a nearest neighbor graph in which their
is an edge between every point and it's $k$ nearest neighbors or points
within $\epsilon$ radius. The geodesic distance between two points is then
assumed to be the shortest path distance between them in the graph so
constructed. The manifold distance matrix is then used to obtain distance
preserving low dimensional representation for the data using
multi-dimensional scaling. The error introduced in reverse-mapping an
Isomap embedding back to the original space is known as the residual
error. Comparing the residual error for Isomap embedding of different
dimensions gives a way of estimating the manifold dimensionality of the
input data \citep{martin05}. We use Isomap to estimate the dimensionality
of the chunk manifolds by comparing the residual variance for embeddings of
different dimension.

% \section{Experiments}
% In this work, we explore some well known design problems with respect to
% the {\em Chunk dimensionality conjecture}. Chapter 2 presents our method of
% extracting and analysing chunks of good design for a multi-objective
% optimization problem, accompanied by its application to the BDCPMM design
% problem introduced in \ref{xbehaviour}. For this problem the design
% implications for the obtained chunks and their corresponence to an expert
% designers intuition are also dicussed. Chapter 3 presents the analysis of
% the well known gearbox design problem \citep{agogino90}. In chapter 4 we
% analyse two small design problems, the clutch brake design and the welded
% beam design problem. The clutch brake design problem is a dicrete
% optimization problem and the welded beam design problem has a continuous
% decision space.
