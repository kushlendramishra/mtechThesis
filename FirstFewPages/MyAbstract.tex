\begin{center}
\begin{large}
{\it{\bf ABSTRACT}}
\end{large}
\end{center} 



For problems in their domain of expertise, human experts are known to
arrive at a few good solutions immediately without exhaustive search. It is
believed that such expertise depends on the learning of optimal chunks that
emerge during the routine exploration of the problem space, constituting a
dimensionally reducing representation for the task domain. Acquiring such
expertise in a task domain is based on discovering functional
interrelations among parameters in a decision space. In most situations,
function is captured through multiple performance objectives, as a result
of which no single optimum may exist.



% Any agent operating in a task domain is expected to acquire some expertise
% in executing repetitive tasks in the domain.  This process of acquiring
% expertise is based on discovering functional interrelations among
% parameters in the decision space.  In this process {\em function} is
% captured through the idea of multiple performance objectives, as a result
% of which no single optimum may exist.  Nonetheless, human experts are known
% to arrive at a few good solutions immediately without exhaustive search,
% while computational algorithms for multi-objective optimization are based
% on search.  It is believed that such expertise depends on the learning of
% optimal ��chunks�� that emerge during the routine exploration of the problem
% space, constituting a dimensionality reducing representation for the task
% domain.

In this work, we propose that the chunks arise as dimensionality-reducing
structures that reflect clusters in the pareto-optimal region. We propose
a {\em Chunk Dimensionality Conjecture}, by which for well-posed objective
functions, the dimensionality of the chunk may be of the order of the
number of objective functions.  We present some empirical studies on this
conjecture through the analysis of several well-known problems in
multi-objective design. We extend the work done in \citep{mukerjee09} to
problems whose pareto-front is not a single continuous manifold, but may
consist of several separate manifolds corresponding to clusters over the
non-dominated space. For this purpose, we develop a variant of DBSCAN that
obtains clusters on the non-dominated solutions obtained from
multi-objective optimization, and illustrate the process of learning chunks
as low-dimensional manifolds for each cluster. The design ramifications of
these chunks are also highlighted for a electromechanical design problem.
 
\vskip 4mm

